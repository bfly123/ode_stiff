\documentclass[review]{elsarticle}
\usepackage{amssymb,lineno}
\modulolinenumbers[5]
\usepackage{color}
\usepackage[colorlinks,linkcolor=black,hyperindex,CJKbookmarks,dvipdfm]{hyperref}
\usepackage{tikz}
rusepackage{graphics,graphicx}
\usepackage{mathrsfs}
\usepackage{amsmath}
\usepackage{subfigure}
\usepackage{booktabs}
\usepackage{bm}
\usepackage{amsthm}
\usepackage{subfigure}
\usepackage{listings}
\usepackage{setspace}
\usepackage[margin=2.5cm]{geometry}
%\usepackage{cite}
\usetikzlibrary{shapes.geometric, arrows,matrix,positioning,calc}
\intextsep=8pt plus 3pt minus 1pt
\tikzstyle{startstop} = [rectangle, rounded corners, minimum width=2cm, minimum height=1cm,text centered, draw=black, fill=red!30]
%\tikzstyle{io} = [trapezium, trapezium left angle=70, trapezium right angle=110, minimum width=3cm, minimum height=1cm, text centered, draw=black, fill=blue!30]
\tikzstyle{process} = [rectangle, minimum width=4cm, minimum height=4cm, text centered, draw=black, fill=orange!30]
\tikzstyle{decision} = [diamond, minimum width=2cm, minimum height=1cm, text centered, draw=black, fill=green!30]
%\%tikzstyle{arrow} = [thick,->,>=stealth]
\theoremstyle{plain}\newtheorem{definition}{\sc{Definition}}
\theoremstyle{defination}\newtheorem{example}{Example}[section]
\definecolor{ColorMark}{rgb}{1,1,1}
\numberwithin{equation}{section}
\numberwithin{table}{section}
\graphicspath{{picture/}} 
\bibliographystyle{elsarticle-num}
\begin{document}


\title{A third-order explicit numerical scheme for stiff ODE equations and it use in **equations }
\author{Li Liu$^1$}
\author{Yiqing Shen$^{1,2}$ \corref{mycorrespondingauthor}}
\cortext[mycorrespondingauthor]{
Correspondence to: Yiqing Shen, LHD, Institute of Mechanics, Chinese Academy of Sciences, Beijing 100190, China. E-mail: yqshen@imech.ac.cn}
\address{$^1$LHD, Institute of Mechanics, Chinese Academy of Sciences, Beijing 100190, China}
\address{$^2$School of Engineering Science, University of Chinese Academy of Sciences, Beijing 100049, China}
{
%\address[mysecondaryaddress]{360 Park Avenue South, New York}
\begin{abstract}
In this paper, we construct a new numerical method to solve the reactive Euler equations to cure the numerical stiffness problem.
First, the species mass equations are decoupled from the reactive Euler equations, and they are further fractionated into the convection step and reaction step.
 In the species convection
step, by introducing two kinds of virtual Lagrangian point (cell-point and particle-point), a dual information preserving (DIP) method is proposed to resolve the convection characteristics. In this new method, the 
 information (including the transport value and the relative location to the centre of current cell) of cell-point and paticle-point are updated according to the velocity field. By using the DIP method, the incorrect activate position of the reaction, which may be caused by the numerical dissipation, can be effectively avoided. In addition, a numerical perturbation method is also developed to solve the fractionated reaction step (ODE equation) to improve the stability and efficiency. A series of numerical examples are presented to validate the accuracy and robustness of the new method. 
\end{abstract}
\begin{keyword}
 Stiff reacting flow\sep  Dual information preserving method\sep Numerical pertutbation method\sep Shock-capturing scheme 
\end{keyword}

\maketitle
\section{Introduction}

The ODE initial value problem
\begin{equation}\label{eq:ode}
  \bm{y}'= \bm{f}(t,\bm{y}(t)), \hspace{0.3cm} \bm{y}(0)= \bm{y}_0,
  \end{equation}
  is considered in this paper. 
  Having a history of over two centuries, developing numerical methods for the ODE equations seems to be an old topic.  However, if ``some components of the solution decay much more rapidly than others''\cite{lambert1991numerical}, the numercial methods will beset by the stiffness, which is still bothering us with stable and efficient problems in many disciplines, for instance in simulating the  chemical kinetics and control theory.

  About in the 1960s, explosion prosperity has happened in the study of the ODE equations. Especially the numerical stability researches done by Dahlquist, Hirshfelder and many other mathematicians, give a more clear glance at the numerical stiffness. Although it is still  difficult to define ``stiffness'' in a precise way, many important theories proposed in that period, such as the famous A-stability\cite{Germund1963A} and the following L-stability\cite{ehle1969pade}, are powerful rulers to measure the stiffness of equations and the stable property of a numerical method. With those theoritic study, a fact is revealed that the explicit one-step methods, for instance, the Runge-Kutta methods, and all the multistep methods cannot be A-stable. Under this background and with the popular use of one-step methods, espcially Runge-Kutta class of methods, nearly a common sense have been achieved, that if we want to solve stiff equations stably  with relatively large steps, we must bear the cost of iteration in an implicit method, for the reason that only implicit methods can achieve both  the A-stability and the  high-order accuracy at the same time.  

  However, this statement may not be true beyond the frames of the  one-step or the multistep methods. In fact, only linear methods have been thoroughly studied but few unlinear ODE methods have been considered. For this reason, some leaners still  hope to construct explicit A-stable methods in a special nonlinear way, which can  break the stablility-barrier of explicit methods and solve the stiff equations more easily. As an example,  Wu\cite{Wu1998A,Wu2000The} constructed a sixth-order A-stable explicit one-step method by adding an exponential term into the traditional Taylor series methods\cite{Barrio2005Performance,Barrio2011Breaking}. In our previous work \cite{Liu2017The}, a third-order A-stable numercal perturbation (NP) method has been proposed and used in solving  the stiff reacting ODE equations. But there is still no universal way to construct A-stable explicit method in every order. 

In this paper, we give a general study of the numerical perturbation method. 
     The main idea of constructing a numerical perturbation method for ODEs is as follows: \textcircled{\small{1}}       




  \subsection{The Taylor series method}

  \begin{equation}
	y_{n+1} = y_n + \frac{dy}{dt}\Delta t +\frac{1}{2} \frac{d^2y}{dt^2}+\cdots
	\end{equation}

\begin{equation}
\begin{align}
  f= Ay\\
  \frac{dy}{dt}=f=Ay\\
  \frac{d^2y}{dt^2}= A^2y 
\end{align}
  \end{equation}


  \begin{equation}
	y_{n+1} = y_n + A y_n \Delta t + A^2\Delta t \frac{1}{2} \frac{d^2y}{dt^2}+\cdots
	\end{equation}


 In this paper, we try to develop a high-order stiff-stable explicit numerical method 

This paper is organized as follows. In section 2, we briefly introduce the decoupling method for solving the reactive Euler equations. In section 3, a dual information preserving method is proposed to solve the convection step of species mass fraction equations. In section 4, a numerical perturbation method is developed to solve the fractionated reaction step, analysis of stability and numerical examples are also presented. A series of examples, including one- and two- dimensional problems, simplified reaction model and multi-species reaction models, are given to validate the accuracy and robustness of the new method in section 5. Conclusions are shown in section 6.
This equations is easy to solve for  


\section{Numerical perturbation method}

For Eq(\ref{eq:ode}), one of  the simplest scheme is the first-order explcit Euler scheme  
\begin{equation}\label{eq:euler}
  \bm{y}_{n+1}- \bm{y}_n = \Delta t f(t,\bm{y})
\end{equation}
If we want to improve the accuracy order of scheme (\ref{eq:euler}),  one common method is add 

But as proved by  

A common way to get higher-order accuracy it to use sub-timestep in the intercal $[t,t+\Delta t]$,    with the stable property in an explicit scheme  is to construct unlinear schemes with more derivative informations from the original differential equation (\ref{eq:ode}) 

First, more dericative information can be get from the original differential equation (\ref{eq:ode}) theoretically 








   



\section{Conclusions}

The dual information preserving method is firstly proposed to cure the numerical stiff problem generated in simulating the reacting flows. First, the species mass fraction equations are decoupled from the reactive Euler equations, and then they are further fractionated into the convection step and reaction step. The DIP method is actually proposed to deal with the species convection step. Two kinds of virtual Lagrangian point are introduced, one is limited in each Eulerian cell, and another one is tracked in the whole computation domain. The number of each kind of virtual point is the same as the cell (grid) number.The DIP method can effectively eliminate the spurious propagation speed caused by the intermediate state generated by the numerical dissipation.

In this paper, the numerical perturbation (NP) methods are also developed to solve the fractionated reaction step (ODE equations). The NP schemes show several advantages, such as no iteration, high order accuracy and large stable region.  

A series of numerical examples are used to demonstrate the reliability and robustness of the new methods.

%^The key idea of this method is
%^
%^(1) the particle information, including the current cell number (particle moved in), relative location
%^to the centre of current cell, and the species mass fraction on the point, is calculated and
%^stored. While the cell information only record the position to the cell centre 
%^and the species mass fraction of the cell.  
%^
%^(2) For a fixed cell, if there are information points in it, its information is
%^updated by averaging all paticle points’ information; otherwise, the cell’s information
%^is updated by averaging all entered cell points’ information. If there is none of above
%^two points in a cell, its relative location of the cell point is set to zero and the species mass on it are obtained by interpolating those of contiguous cells. 
%^
%By the new method, the correct activate position of the reaction will be captured even with a course mesh as eliminating the dissapation in the species mass fraction equations. The DIP method is easily extended to multiple reaction models and multi-dimensional problems. Through decoupling process, this method can combine with high order methods for Euler equations system, conveniently. In this paper, numerical perturbation method is also developed to solve the fractioned reaction step (ODE equations) to improve the stability and efficiency. A series of numerical examples are used to test the reliability and robustness of the new methods.  
%In the furture, the DIP method may also be used in traditional interface(surface) problems and reaction flows with more complex reaction models.
%


%In addition, a numerical perturbation method to solve the fractioned reaction step (ODE equation) to improve the stability and
%efficiency.





%we developed a series of high-order numerical perturbation method for stiff ODE equations and Dual Information Preserving(DIP) method for mass fraction convection equations to capture the reaction interface. Combine them and use a new decoupling method we gave a solving system for reactive Euler equations with high order. We can cure the difficulty in numerical simulations of the conservation equations with stiff source terms by the new solving system.  

%With a decoupling method we decouple reactive Euler equations into two parts---Euler equations and mass fraction convection/reaction equations, Euler equations can be solved by fifth-order WENO/third-order Runge-Kutta schemes with a Lax-Friedrichs splitting. Then using fraction step method, we split mass fraction equations into two steps: convection step and reaction step. Third-order transformed perturbation(TP) method is used in solving reaction step ODE equations which is proved to be strong A-stable. Convection step is solved by Dual Information Preserving(DIP) method which uses Lagrangian points based on fixed mesh. Eliminating unphysical smooth in the mass fraction discontinuous caused by the dissipation of spatial difference schemes in convection step, the wrong activation of reaction will not happened. Through a lot numerical examples, includes scalar problems and reactive Euler problems, one- and two- dimension detonation problems, single reaction models and multiple reaction models, the availability and robustness of the new method is tested.

\section*{Acknowledgement} 
This research work was supported by NSFC 11272324, 11272325, NSAF U1530145 and 2016YFA0401200.

\section*{References}

\bibliography{mybibfile}

\end{document}






\end{note}


